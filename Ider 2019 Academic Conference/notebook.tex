
% Default to the notebook output style

    


% Inherit from the specified cell style.




    
\documentclass[11pt]{article}

    
    
    \usepackage[T1]{fontenc}
    % Nicer default font (+ math font) than Computer Modern for most use cases
    \usepackage{mathpazo}

    % Basic figure setup, for now with no caption control since it's done
    % automatically by Pandoc (which extracts ![](path) syntax from Markdown).
    \usepackage{graphicx}
    % We will generate all images so they have a width \maxwidth. This means
    % that they will get their normal width if they fit onto the page, but
    % are scaled down if they would overflow the margins.
    \makeatletter
    \def\maxwidth{\ifdim\Gin@nat@width>\linewidth\linewidth
    \else\Gin@nat@width\fi}
    \makeatother
    \let\Oldincludegraphics\includegraphics
    % Set max figure width to be 80% of text width, for now hardcoded.
    \renewcommand{\includegraphics}[1]{\Oldincludegraphics[width=.8\maxwidth]{#1}}
    % Ensure that by default, figures have no caption (until we provide a
    % proper Figure object with a Caption API and a way to capture that
    % in the conversion process - todo).
    \usepackage{caption}
    \DeclareCaptionLabelFormat{nolabel}{}
    \captionsetup{labelformat=nolabel}

    \usepackage{adjustbox} % Used to constrain images to a maximum size 
    \usepackage{xcolor} % Allow colors to be defined
    \usepackage{enumerate} % Needed for markdown enumerations to work
    \usepackage{geometry} % Used to adjust the document margins
    \usepackage{amsmath} % Equations
    \usepackage{amssymb} % Equations
    \usepackage{textcomp} % defines textquotesingle
    % Hack from http://tex.stackexchange.com/a/47451/13684:
    \AtBeginDocument{%
        \def\PYZsq{\textquotesingle}% Upright quotes in Pygmentized code
    }
    \usepackage{upquote} % Upright quotes for verbatim code
    \usepackage{eurosym} % defines \euro
    \usepackage[mathletters]{ucs} % Extended unicode (utf-8) support
    \usepackage[utf8x]{inputenc} % Allow utf-8 characters in the tex document
    \usepackage{fancyvrb} % verbatim replacement that allows latex
    \usepackage{grffile} % extends the file name processing of package graphics 
                         % to support a larger range 
    % The hyperref package gives us a pdf with properly built
    % internal navigation ('pdf bookmarks' for the table of contents,
    % internal cross-reference links, web links for URLs, etc.)
    \usepackage{hyperref}
    \usepackage{longtable} % longtable support required by pandoc >1.10
    \usepackage{booktabs}  % table support for pandoc > 1.12.2
    \usepackage[inline]{enumitem} % IRkernel/repr support (it uses the enumerate* environment)
    \usepackage[normalem]{ulem} % ulem is needed to support strikethroughs (\sout)
                                % normalem makes italics be italics, not underlines
    \usepackage{mathrsfs}
    

    
    
    % Colors for the hyperref package
    \definecolor{urlcolor}{rgb}{0,.145,.698}
    \definecolor{linkcolor}{rgb}{.71,0.21,0.01}
    \definecolor{citecolor}{rgb}{.12,.54,.11}

    % ANSI colors
    \definecolor{ansi-black}{HTML}{3E424D}
    \definecolor{ansi-black-intense}{HTML}{282C36}
    \definecolor{ansi-red}{HTML}{E75C58}
    \definecolor{ansi-red-intense}{HTML}{B22B31}
    \definecolor{ansi-green}{HTML}{00A250}
    \definecolor{ansi-green-intense}{HTML}{007427}
    \definecolor{ansi-yellow}{HTML}{DDB62B}
    \definecolor{ansi-yellow-intense}{HTML}{B27D12}
    \definecolor{ansi-blue}{HTML}{208FFB}
    \definecolor{ansi-blue-intense}{HTML}{0065CA}
    \definecolor{ansi-magenta}{HTML}{D160C4}
    \definecolor{ansi-magenta-intense}{HTML}{A03196}
    \definecolor{ansi-cyan}{HTML}{60C6C8}
    \definecolor{ansi-cyan-intense}{HTML}{258F8F}
    \definecolor{ansi-white}{HTML}{C5C1B4}
    \definecolor{ansi-white-intense}{HTML}{A1A6B2}
    \definecolor{ansi-default-inverse-fg}{HTML}{FFFFFF}
    \definecolor{ansi-default-inverse-bg}{HTML}{000000}

    % commands and environments needed by pandoc snippets
    % extracted from the output of `pandoc -s`
    \providecommand{\tightlist}{%
      \setlength{\itemsep}{0pt}\setlength{\parskip}{0pt}}
    \DefineVerbatimEnvironment{Highlighting}{Verbatim}{commandchars=\\\{\}}
    % Add ',fontsize=\small' for more characters per line
    \newenvironment{Shaded}{}{}
    \newcommand{\KeywordTok}[1]{\textcolor[rgb]{0.00,0.44,0.13}{\textbf{{#1}}}}
    \newcommand{\DataTypeTok}[1]{\textcolor[rgb]{0.56,0.13,0.00}{{#1}}}
    \newcommand{\DecValTok}[1]{\textcolor[rgb]{0.25,0.63,0.44}{{#1}}}
    \newcommand{\BaseNTok}[1]{\textcolor[rgb]{0.25,0.63,0.44}{{#1}}}
    \newcommand{\FloatTok}[1]{\textcolor[rgb]{0.25,0.63,0.44}{{#1}}}
    \newcommand{\CharTok}[1]{\textcolor[rgb]{0.25,0.44,0.63}{{#1}}}
    \newcommand{\StringTok}[1]{\textcolor[rgb]{0.25,0.44,0.63}{{#1}}}
    \newcommand{\CommentTok}[1]{\textcolor[rgb]{0.38,0.63,0.69}{\textit{{#1}}}}
    \newcommand{\OtherTok}[1]{\textcolor[rgb]{0.00,0.44,0.13}{{#1}}}
    \newcommand{\AlertTok}[1]{\textcolor[rgb]{1.00,0.00,0.00}{\textbf{{#1}}}}
    \newcommand{\FunctionTok}[1]{\textcolor[rgb]{0.02,0.16,0.49}{{#1}}}
    \newcommand{\RegionMarkerTok}[1]{{#1}}
    \newcommand{\ErrorTok}[1]{\textcolor[rgb]{1.00,0.00,0.00}{\textbf{{#1}}}}
    \newcommand{\NormalTok}[1]{{#1}}
    
    % Additional commands for more recent versions of Pandoc
    \newcommand{\ConstantTok}[1]{\textcolor[rgb]{0.53,0.00,0.00}{{#1}}}
    \newcommand{\SpecialCharTok}[1]{\textcolor[rgb]{0.25,0.44,0.63}{{#1}}}
    \newcommand{\VerbatimStringTok}[1]{\textcolor[rgb]{0.25,0.44,0.63}{{#1}}}
    \newcommand{\SpecialStringTok}[1]{\textcolor[rgb]{0.73,0.40,0.53}{{#1}}}
    \newcommand{\ImportTok}[1]{{#1}}
    \newcommand{\DocumentationTok}[1]{\textcolor[rgb]{0.73,0.13,0.13}{\textit{{#1}}}}
    \newcommand{\AnnotationTok}[1]{\textcolor[rgb]{0.38,0.63,0.69}{\textbf{\textit{{#1}}}}}
    \newcommand{\CommentVarTok}[1]{\textcolor[rgb]{0.38,0.63,0.69}{\textbf{\textit{{#1}}}}}
    \newcommand{\VariableTok}[1]{\textcolor[rgb]{0.10,0.09,0.49}{{#1}}}
    \newcommand{\ControlFlowTok}[1]{\textcolor[rgb]{0.00,0.44,0.13}{\textbf{{#1}}}}
    \newcommand{\OperatorTok}[1]{\textcolor[rgb]{0.40,0.40,0.40}{{#1}}}
    \newcommand{\BuiltInTok}[1]{{#1}}
    \newcommand{\ExtensionTok}[1]{{#1}}
    \newcommand{\PreprocessorTok}[1]{\textcolor[rgb]{0.74,0.48,0.00}{{#1}}}
    \newcommand{\AttributeTok}[1]{\textcolor[rgb]{0.49,0.56,0.16}{{#1}}}
    \newcommand{\InformationTok}[1]{\textcolor[rgb]{0.38,0.63,0.69}{\textbf{\textit{{#1}}}}}
    \newcommand{\WarningTok}[1]{\textcolor[rgb]{0.38,0.63,0.69}{\textbf{\textit{{#1}}}}}
    
    
    % Define a nice break command that doesn't care if a line doesn't already
    % exist.
    \def\br{\hspace*{\fill} \\* }
    % Math Jax compatibility definitions
    \def\gt{>}
    \def\lt{<}
    \let\Oldtex\TeX
    \let\Oldlatex\LaTeX
    \renewcommand{\TeX}{\textrm{\Oldtex}}
    \renewcommand{\LaTeX}{\textrm{\Oldlatex}}
    % Document parameters
    % Document title
    \title{Data Storytelling in Research}
    
    
    
    
    

    % Pygments definitions
    
\makeatletter
\def\PY@reset{\let\PY@it=\relax \let\PY@bf=\relax%
    \let\PY@ul=\relax \let\PY@tc=\relax%
    \let\PY@bc=\relax \let\PY@ff=\relax}
\def\PY@tok#1{\csname PY@tok@#1\endcsname}
\def\PY@toks#1+{\ifx\relax#1\empty\else%
    \PY@tok{#1}\expandafter\PY@toks\fi}
\def\PY@do#1{\PY@bc{\PY@tc{\PY@ul{%
    \PY@it{\PY@bf{\PY@ff{#1}}}}}}}
\def\PY#1#2{\PY@reset\PY@toks#1+\relax+\PY@do{#2}}

\expandafter\def\csname PY@tok@w\endcsname{\def\PY@tc##1{\textcolor[rgb]{0.73,0.73,0.73}{##1}}}
\expandafter\def\csname PY@tok@c\endcsname{\let\PY@it=\textit\def\PY@tc##1{\textcolor[rgb]{0.25,0.50,0.50}{##1}}}
\expandafter\def\csname PY@tok@cp\endcsname{\def\PY@tc##1{\textcolor[rgb]{0.74,0.48,0.00}{##1}}}
\expandafter\def\csname PY@tok@k\endcsname{\let\PY@bf=\textbf\def\PY@tc##1{\textcolor[rgb]{0.00,0.50,0.00}{##1}}}
\expandafter\def\csname PY@tok@kp\endcsname{\def\PY@tc##1{\textcolor[rgb]{0.00,0.50,0.00}{##1}}}
\expandafter\def\csname PY@tok@kt\endcsname{\def\PY@tc##1{\textcolor[rgb]{0.69,0.00,0.25}{##1}}}
\expandafter\def\csname PY@tok@o\endcsname{\def\PY@tc##1{\textcolor[rgb]{0.40,0.40,0.40}{##1}}}
\expandafter\def\csname PY@tok@ow\endcsname{\let\PY@bf=\textbf\def\PY@tc##1{\textcolor[rgb]{0.67,0.13,1.00}{##1}}}
\expandafter\def\csname PY@tok@nb\endcsname{\def\PY@tc##1{\textcolor[rgb]{0.00,0.50,0.00}{##1}}}
\expandafter\def\csname PY@tok@nf\endcsname{\def\PY@tc##1{\textcolor[rgb]{0.00,0.00,1.00}{##1}}}
\expandafter\def\csname PY@tok@nc\endcsname{\let\PY@bf=\textbf\def\PY@tc##1{\textcolor[rgb]{0.00,0.00,1.00}{##1}}}
\expandafter\def\csname PY@tok@nn\endcsname{\let\PY@bf=\textbf\def\PY@tc##1{\textcolor[rgb]{0.00,0.00,1.00}{##1}}}
\expandafter\def\csname PY@tok@ne\endcsname{\let\PY@bf=\textbf\def\PY@tc##1{\textcolor[rgb]{0.82,0.25,0.23}{##1}}}
\expandafter\def\csname PY@tok@nv\endcsname{\def\PY@tc##1{\textcolor[rgb]{0.10,0.09,0.49}{##1}}}
\expandafter\def\csname PY@tok@no\endcsname{\def\PY@tc##1{\textcolor[rgb]{0.53,0.00,0.00}{##1}}}
\expandafter\def\csname PY@tok@nl\endcsname{\def\PY@tc##1{\textcolor[rgb]{0.63,0.63,0.00}{##1}}}
\expandafter\def\csname PY@tok@ni\endcsname{\let\PY@bf=\textbf\def\PY@tc##1{\textcolor[rgb]{0.60,0.60,0.60}{##1}}}
\expandafter\def\csname PY@tok@na\endcsname{\def\PY@tc##1{\textcolor[rgb]{0.49,0.56,0.16}{##1}}}
\expandafter\def\csname PY@tok@nt\endcsname{\let\PY@bf=\textbf\def\PY@tc##1{\textcolor[rgb]{0.00,0.50,0.00}{##1}}}
\expandafter\def\csname PY@tok@nd\endcsname{\def\PY@tc##1{\textcolor[rgb]{0.67,0.13,1.00}{##1}}}
\expandafter\def\csname PY@tok@s\endcsname{\def\PY@tc##1{\textcolor[rgb]{0.73,0.13,0.13}{##1}}}
\expandafter\def\csname PY@tok@sd\endcsname{\let\PY@it=\textit\def\PY@tc##1{\textcolor[rgb]{0.73,0.13,0.13}{##1}}}
\expandafter\def\csname PY@tok@si\endcsname{\let\PY@bf=\textbf\def\PY@tc##1{\textcolor[rgb]{0.73,0.40,0.53}{##1}}}
\expandafter\def\csname PY@tok@se\endcsname{\let\PY@bf=\textbf\def\PY@tc##1{\textcolor[rgb]{0.73,0.40,0.13}{##1}}}
\expandafter\def\csname PY@tok@sr\endcsname{\def\PY@tc##1{\textcolor[rgb]{0.73,0.40,0.53}{##1}}}
\expandafter\def\csname PY@tok@ss\endcsname{\def\PY@tc##1{\textcolor[rgb]{0.10,0.09,0.49}{##1}}}
\expandafter\def\csname PY@tok@sx\endcsname{\def\PY@tc##1{\textcolor[rgb]{0.00,0.50,0.00}{##1}}}
\expandafter\def\csname PY@tok@m\endcsname{\def\PY@tc##1{\textcolor[rgb]{0.40,0.40,0.40}{##1}}}
\expandafter\def\csname PY@tok@gh\endcsname{\let\PY@bf=\textbf\def\PY@tc##1{\textcolor[rgb]{0.00,0.00,0.50}{##1}}}
\expandafter\def\csname PY@tok@gu\endcsname{\let\PY@bf=\textbf\def\PY@tc##1{\textcolor[rgb]{0.50,0.00,0.50}{##1}}}
\expandafter\def\csname PY@tok@gd\endcsname{\def\PY@tc##1{\textcolor[rgb]{0.63,0.00,0.00}{##1}}}
\expandafter\def\csname PY@tok@gi\endcsname{\def\PY@tc##1{\textcolor[rgb]{0.00,0.63,0.00}{##1}}}
\expandafter\def\csname PY@tok@gr\endcsname{\def\PY@tc##1{\textcolor[rgb]{1.00,0.00,0.00}{##1}}}
\expandafter\def\csname PY@tok@ge\endcsname{\let\PY@it=\textit}
\expandafter\def\csname PY@tok@gs\endcsname{\let\PY@bf=\textbf}
\expandafter\def\csname PY@tok@gp\endcsname{\let\PY@bf=\textbf\def\PY@tc##1{\textcolor[rgb]{0.00,0.00,0.50}{##1}}}
\expandafter\def\csname PY@tok@go\endcsname{\def\PY@tc##1{\textcolor[rgb]{0.53,0.53,0.53}{##1}}}
\expandafter\def\csname PY@tok@gt\endcsname{\def\PY@tc##1{\textcolor[rgb]{0.00,0.27,0.87}{##1}}}
\expandafter\def\csname PY@tok@err\endcsname{\def\PY@bc##1{\setlength{\fboxsep}{0pt}\fcolorbox[rgb]{1.00,0.00,0.00}{1,1,1}{\strut ##1}}}
\expandafter\def\csname PY@tok@kc\endcsname{\let\PY@bf=\textbf\def\PY@tc##1{\textcolor[rgb]{0.00,0.50,0.00}{##1}}}
\expandafter\def\csname PY@tok@kd\endcsname{\let\PY@bf=\textbf\def\PY@tc##1{\textcolor[rgb]{0.00,0.50,0.00}{##1}}}
\expandafter\def\csname PY@tok@kn\endcsname{\let\PY@bf=\textbf\def\PY@tc##1{\textcolor[rgb]{0.00,0.50,0.00}{##1}}}
\expandafter\def\csname PY@tok@kr\endcsname{\let\PY@bf=\textbf\def\PY@tc##1{\textcolor[rgb]{0.00,0.50,0.00}{##1}}}
\expandafter\def\csname PY@tok@bp\endcsname{\def\PY@tc##1{\textcolor[rgb]{0.00,0.50,0.00}{##1}}}
\expandafter\def\csname PY@tok@fm\endcsname{\def\PY@tc##1{\textcolor[rgb]{0.00,0.00,1.00}{##1}}}
\expandafter\def\csname PY@tok@vc\endcsname{\def\PY@tc##1{\textcolor[rgb]{0.10,0.09,0.49}{##1}}}
\expandafter\def\csname PY@tok@vg\endcsname{\def\PY@tc##1{\textcolor[rgb]{0.10,0.09,0.49}{##1}}}
\expandafter\def\csname PY@tok@vi\endcsname{\def\PY@tc##1{\textcolor[rgb]{0.10,0.09,0.49}{##1}}}
\expandafter\def\csname PY@tok@vm\endcsname{\def\PY@tc##1{\textcolor[rgb]{0.10,0.09,0.49}{##1}}}
\expandafter\def\csname PY@tok@sa\endcsname{\def\PY@tc##1{\textcolor[rgb]{0.73,0.13,0.13}{##1}}}
\expandafter\def\csname PY@tok@sb\endcsname{\def\PY@tc##1{\textcolor[rgb]{0.73,0.13,0.13}{##1}}}
\expandafter\def\csname PY@tok@sc\endcsname{\def\PY@tc##1{\textcolor[rgb]{0.73,0.13,0.13}{##1}}}
\expandafter\def\csname PY@tok@dl\endcsname{\def\PY@tc##1{\textcolor[rgb]{0.73,0.13,0.13}{##1}}}
\expandafter\def\csname PY@tok@s2\endcsname{\def\PY@tc##1{\textcolor[rgb]{0.73,0.13,0.13}{##1}}}
\expandafter\def\csname PY@tok@sh\endcsname{\def\PY@tc##1{\textcolor[rgb]{0.73,0.13,0.13}{##1}}}
\expandafter\def\csname PY@tok@s1\endcsname{\def\PY@tc##1{\textcolor[rgb]{0.73,0.13,0.13}{##1}}}
\expandafter\def\csname PY@tok@mb\endcsname{\def\PY@tc##1{\textcolor[rgb]{0.40,0.40,0.40}{##1}}}
\expandafter\def\csname PY@tok@mf\endcsname{\def\PY@tc##1{\textcolor[rgb]{0.40,0.40,0.40}{##1}}}
\expandafter\def\csname PY@tok@mh\endcsname{\def\PY@tc##1{\textcolor[rgb]{0.40,0.40,0.40}{##1}}}
\expandafter\def\csname PY@tok@mi\endcsname{\def\PY@tc##1{\textcolor[rgb]{0.40,0.40,0.40}{##1}}}
\expandafter\def\csname PY@tok@il\endcsname{\def\PY@tc##1{\textcolor[rgb]{0.40,0.40,0.40}{##1}}}
\expandafter\def\csname PY@tok@mo\endcsname{\def\PY@tc##1{\textcolor[rgb]{0.40,0.40,0.40}{##1}}}
\expandafter\def\csname PY@tok@ch\endcsname{\let\PY@it=\textit\def\PY@tc##1{\textcolor[rgb]{0.25,0.50,0.50}{##1}}}
\expandafter\def\csname PY@tok@cm\endcsname{\let\PY@it=\textit\def\PY@tc##1{\textcolor[rgb]{0.25,0.50,0.50}{##1}}}
\expandafter\def\csname PY@tok@cpf\endcsname{\let\PY@it=\textit\def\PY@tc##1{\textcolor[rgb]{0.25,0.50,0.50}{##1}}}
\expandafter\def\csname PY@tok@c1\endcsname{\let\PY@it=\textit\def\PY@tc##1{\textcolor[rgb]{0.25,0.50,0.50}{##1}}}
\expandafter\def\csname PY@tok@cs\endcsname{\let\PY@it=\textit\def\PY@tc##1{\textcolor[rgb]{0.25,0.50,0.50}{##1}}}

\def\PYZbs{\char`\\}
\def\PYZus{\char`\_}
\def\PYZob{\char`\{}
\def\PYZcb{\char`\}}
\def\PYZca{\char`\^}
\def\PYZam{\char`\&}
\def\PYZlt{\char`\<}
\def\PYZgt{\char`\>}
\def\PYZsh{\char`\#}
\def\PYZpc{\char`\%}
\def\PYZdl{\char`\$}
\def\PYZhy{\char`\-}
\def\PYZsq{\char`\'}
\def\PYZdq{\char`\"}
\def\PYZti{\char`\~}
% for compatibility with earlier versions
\def\PYZat{@}
\def\PYZlb{[}
\def\PYZrb{]}
\makeatother


    % Exact colors from NB
    \definecolor{incolor}{rgb}{0.0, 0.0, 0.5}
    \definecolor{outcolor}{rgb}{0.545, 0.0, 0.0}



    
    % Prevent overflowing lines due to hard-to-break entities
    \sloppy 
    % Setup hyperref package
    \hypersetup{
      breaklinks=true,  % so long urls are correctly broken across lines
      colorlinks=true,
      urlcolor=urlcolor,
      linkcolor=linkcolor,
      citecolor=citecolor,
      }
    % Slightly bigger margins than the latex defaults
    
    \geometry{verbose,tmargin=1in,bmargin=1in,lmargin=1in,rmargin=1in}
    
    

    \begin{document}
    
    
    \maketitle
    
    

    
    \hypertarget{data-storytelling-in-research}{%
\section{Data Storytelling in
Research}\label{data-storytelling-in-research}}

\hypertarget{robert-ritz}{%
\paragraph{\texorpdfstring{\textbf{Robert
Ritz}}{Robert Ritz}}\label{robert-ritz}}

Data Scientist\\
Director, LETU Mongolia\\
robert@letumongolia.com\\
robertritz@ider.edu.mn\\
github.com/robertritz

    

    

    \hypertarget{issues-with-pollution-prediction}{%
\subsection{Issues With Pollution
Prediction}\label{issues-with-pollution-prediction}}

\begin{itemize}
\tightlist
\item
  Do people actually want to know the forecast?
\item
  What do these AQ categories mean?
\item
  Air quality is bad, so what?
\end{itemize}

    Data Science and its multidisciplinary approach is very similar to the
approach journalists must take to tackle data journalism. However,
domain knowledge in the form of journalistic experience is often not
enough.

    \hypertarget{drew-conways-data-science-venn-diagram}{%
\subsection{Drew Conway's Data Science Venn
Diagram}\label{drew-conways-data-science-venn-diagram}}

    \hypertarget{data-analysis-process}{%
\subsection{Data Analysis Process}\label{data-analysis-process}}

\begin{enumerate}
\def\labelenumi{\arabic{enumi}.}
\tightlist
\item
  Define problem and scope
\item
  Find data
\item
  Clean, transform data
\item
  Exploratory data analysis (EDA)
\item
  Communicate and interpret results
\end{enumerate}

    \hypertarget{problems-with-academic-research-process}{%
\subsection{Problems with academic research
process}\label{problems-with-academic-research-process}}

\begin{itemize}
\tightlist
\item
  Too much focus on process
\item
  Too little focus on communication
\item
  Rigid rules for interpretation
\item
  Lack of focus on impact
\end{itemize}

    \hypertarget{some-of-my-work}{%
\section{Some of My Work}\label{some-of-my-work}}

    \hypertarget{chronic-youth-unemployment}{%
\section{Chronic Youth Unemployment}\label{chronic-youth-unemployment}}

\hypertarget{hypothesis-unemployment-is-distributed-evenly-throughout-the-population.}{%
\subsection{Hypothesis: Unemployment is distributed evenly throughout
the
population.}\label{hypothesis-unemployment-is-distributed-evenly-throughout-the-population.}}

    \hypertarget{data-1212.mn}{%
\subsection{Data: 1212.mn}\label{data-1212.mn}}

\begin{itemize}
\tightlist
\item
  UNEMPLOYED RATE, by sex, age group
\item
  38 different tables in the \textbf{Labour force, unemployment}
  category
\end{itemize}

    \hypertarget{challenge-data-isnt-always-in-the-right-format}{%
\subsection{Challenge: Data isn't always in the right
format}\label{challenge-data-isnt-always-in-the-right-format}}

    \hypertarget{tool-powerquery-in-excel}{%
\subsection{Tool: PowerQuery in Excel}\label{tool-powerquery-in-excel}}

\begin{itemize}
\tightlist
\item
  Alternatives: Python, R, OpenRefine
\end{itemize}

    

    \hypertarget{using-pandas-to-import-the-data}{%
\subsection{Using Pandas to import the
data}\label{using-pandas-to-import-the-data}}

    \begin{Verbatim}[commandchars=\\\{\}]
{\color{incolor}In [{\color{incolor}17}]:} \PY{k+kn}{import} \PY{n+nn}{pandas} \PY{k}{as} \PY{n+nn}{pd}
         \PY{n}{unemployment} \PY{o}{=} \PY{n}{pd}\PY{o}{.}\PY{n}{read\PYZus{}csv}\PY{p}{(}\PY{l+s+s2}{\PYZdq{}}\PY{l+s+s2}{assets/1212\PYZhy{}unemployment\PYZhy{}gender.csv}\PY{l+s+s2}{\PYZdq{}}\PY{p}{)}
         \PY{n}{unemployment}\PY{o}{.}\PY{n}{head}\PY{p}{(}\PY{p}{)}
\end{Verbatim}

\begin{Verbatim}[commandchars=\\\{\}]
{\color{outcolor}Out[{\color{outcolor}17}]:}   Age group     Sex  2009  2010  2011  2012  2013  2014  2015  2016  2017
         0     Total   Total  11.6   9.9   7.7   8.2   7.9   7.9   7.5  10.0   8.8
         1     Total    Male  11.6  10.5   8.1   8.4   7.6   8.5   8.2  11.6   9.6
         2     Total  Female  11.6   9.2   7.4   8.1   8.3   7.3   6.7   8.2   7.8
         3     15-19   Total  22.8  18.0  14.0   8.0  13.6  17.4  18.8  24.6  24.8
         4     15-19    Male  23.1  17.5  14.6   6.9  11.1  16.2  17.4  22.2  18.8
\end{Verbatim}
            
    \hypertarget{unpivot-by-melting}{%
\subsection{Unpivot by melting}\label{unpivot-by-melting}}

    \begin{Verbatim}[commandchars=\\\{\}]
{\color{incolor}In [{\color{incolor}16}]:} \PY{n}{unemployment\PYZus{}unpivot} \PY{o}{=} \PY{n}{pd}\PY{o}{.}\PY{n}{melt}\PY{p}{(}\PY{n}{unemployment}\PY{p}{,} \PY{n}{id\PYZus{}vars}\PY{o}{=}\PY{p}{[}\PY{l+s+s1}{\PYZsq{}}\PY{l+s+s1}{Age group}\PY{l+s+s1}{\PYZsq{}}\PY{p}{,} \PY{l+s+s1}{\PYZsq{}}\PY{l+s+s1}{Sex}\PY{l+s+s1}{\PYZsq{}}\PY{p}{]}\PY{p}{,} \PY{n}{var\PYZus{}name}\PY{o}{=}\PY{p}{[}\PY{l+s+s1}{\PYZsq{}}\PY{l+s+s1}{year}\PY{l+s+s1}{\PYZsq{}}\PY{p}{]}\PY{p}{)}
         \PY{n}{unemployment\PYZus{}unpivot}\PY{o}{.}\PY{n}{head}\PY{p}{(}\PY{l+m+mi}{10}\PY{p}{)}
\end{Verbatim}

\begin{Verbatim}[commandchars=\\\{\}]
{\color{outcolor}Out[{\color{outcolor}16}]:}   Age group     Sex  year  value
         0     Total   Total  2009   11.6
         1     Total    Male  2009   11.6
         2     Total  Female  2009   11.6
         3     15-19   Total  2009   22.8
         4     15-19    Male  2009   23.1
         5     15-19  Female  2009   22.4
         6     20-24   Total  2009   21.7
         7     20-24    Male  2009   21.0
         8     20-24  Female  2009   22.6
         9     25-29   Total  2009   12.2
\end{Verbatim}
            
    \hypertarget{after-some-digging-and-playing-around}{%
\section{After some digging and playing
around\ldots{}}\label{after-some-digging-and-playing-around}}

    

    \hypertarget{tools-used}{%
\subsection{Tools Used}\label{tools-used}}

\begin{itemize}
\tightlist
\item
  Excel for cleansing and EDA
\item
  Data Illustrator

  \begin{itemize}
  \tightlist
  \item
    Joint venture by Adobe and Georgia Tech
  \item
    Balances ease of use like Tableau with the customizability of Adobe
    Illustrator
  \end{itemize}
\end{itemize}

    

    \hypertarget{goats-are-taking-over-the-steppe}{%
\section{Goats are Taking Over the
Steppe}\label{goats-are-taking-over-the-steppe}}

    \hypertarget{hypothesis-herd-size-and-composition-has-had-a-large-impact-on-mongolian-rangeland.}{%
\subsection{Hypothesis: Herd size and composition has had a large impact
on Mongolian
rangeland.}\label{hypothesis-herd-size-and-composition-has-had-a-large-impact-on-mongolian-rangeland.}}

\begin{itemize}
\tightlist
\item
  Is climate change the only thing responsible for land degradation?
\item
  Are herders better off today than before?
\end{itemize}

    \hypertarget{data-1212.mn}{%
\subsection{\#\# Data: 1212.mn}\label{data-1212.mn}}

NUMBER OF LIVESTOCK, by type, by region, soum, aimag and the Capital

    

    

    \hypertarget{challenges-and-additional-questions}{%
\subsection{Challenges and additional
questions}\label{challenges-and-additional-questions}}

\begin{itemize}
\tightlist
\item
  How do we know data before 1991 was accurate?
\item
  If herd numbers are growing so much, why is there so much migration to
  UB?
\item
  Goats are the problem\ldots{}..right?
\end{itemize}

    \begin{itemize}
\tightlist
\item
  What more data do we need to answer our new questions?
\item
  Ulaanbaatar migration data. How many people move to UB from the
  countryside?
\item
  UB migration ban until 2020
\item
  We can't analyze what we can't measure.
\end{itemize}

    \hypertarget{crime-dashboard}{%
\section{Crime Dashboard}\label{crime-dashboard}}

\hypertarget{data-httpcrimemap.police.gov.mn}{%
\subsection{Data:
http://crimemap.police.gov.mn}\label{data-httpcrimemap.police.gov.mn}}

\begin{itemize}
\tightlist
\item
  2015 and 2016 appeared complete
\item
  2013, 2014, and 2017 seem to be missing a lot of data
\end{itemize}

    

    \hypertarget{hypothesis-mongolia-has-a-lower-murder-rate-than-the-united-states.}{%
\subsection{Hypothesis: Mongolia has a lower murder rate than the United
States.}\label{hypothesis-mongolia-has-a-lower-murder-rate-than-the-united-states.}}

\begin{itemize}
\tightlist
\item
  Is the murder rate (murders/100,000 people) higher or lower than the
  US?
\end{itemize}

    \hypertarget{poll-in-2016-was-the-mongolian-murder-rate-higher-than-the-united-states-murder-rate}{%
\subsection{Poll: In 2016 was the Mongolian murder rate higher than the
United States murder
rate?}\label{poll-in-2016-was-the-mongolian-murder-rate-higher-than-the-united-states-murder-rate}}

    \hypertarget{mongolia-murder-rate-5.72-5.66-according-to-un-cts-2016-us-murder-rate-5.35---note-un-cts-reported-171-murders-in-2016.-crimemap-reported-173.}{%
\subsection{\texorpdfstring{2016 Mongolia murder rate\emph{: 5.72 (5.66
according to UN-CTS) \#\# 2016 US murder rate: 5.35 - Note}: UN-CTS
reported 171 murders in 2016. Crimemap reported
173.}{2016 Mongolia murder rate: 5.72 (5.66 according to UN-CTS) \#\# 2016 US murder rate: 5.35 - Note: UN-CTS reported 171 murders in 2016. Crimemap reported 173.}}\label{mongolia-murder-rate-5.72-5.66-according-to-un-cts-2016-us-murder-rate-5.35---note-un-cts-reported-171-murders-in-2016.-crimemap-reported-173.}}

\begin{itemize}
\tightlist
\item
  Source: UN-CTS and crimemap.police.gov.mn
\item
  2016 data became available in June this year.
\end{itemize}

    \hypertarget{google-data-studio}{%
\subsection{Google Data Studio}\label{google-data-studio}}

\begin{itemize}
\tightlist
\item
  Crime data scraped from crimemap.police.gov.mn
\item
  Combined population statistics with crime data
\item
  Tool was originally meant for YouTube channel analytics
\end{itemize}

    

    

    \hypertarget{challenges}{%
\subsection{Challenges}\label{challenges}}

\begin{itemize}
\tightlist
\item
  Limited in what we can analyze from this data.

  \begin{itemize}
  \tightlist
  \item
    Lack of detailed information about crime.
  \item
    Who was the victim? Theft has four categories, assault has one.
  \end{itemize}
\item
  Was it closed or not? Did it lead to a prosecution?
\item
  Data doesn't seem to be added consistently after 2016.
\item
  Only works 50\% of the time.
\end{itemize}

    \hypertarget{outstanding-loans}{%
\section{Outstanding Loans}\label{outstanding-loans}}

\hypertarget{hypothesis-debt-is-becoming-unsustainable-in-mongolia.}{%
\subsection{Hypothesis: Debt is becoming unsustainable in
Mongolia.}\label{hypothesis-debt-is-becoming-unsustainable-in-mongolia.}}

\begin{itemize}
\tightlist
\item
  How is debt spread around the country?
\item
  How are deposits and outstanding loans correlated?
\end{itemize}

    \hypertarget{data-1212.mn}{%
\subsection{Data: 1212.mn}\label{data-1212.mn}}

\hypertarget{tool-tableau}{%
\subsection{Tool: Tableau}\label{tool-tableau}}

    

    \hypertarget{challenge}{%
\subsection{Challenge}\label{challenge}}

\begin{itemize}
\tightlist
\item
  Map outstanding loans by aimag
\item
  Raw numbers didn't feel enough
\item
  Per capita (outstanding loans per person)
\end{itemize}

    

    \hypertarget{challenge}{%
\subsection{Challenge:}\label{challenge}}

\begin{itemize}
\tightlist
\item
  What do these numbers mean?
\item
  These are not disaggregated by private or business loans.
\item
  Ulaanbaatar is the economic center. Are banks unwilling to give loans
  for countryside citizens?
\end{itemize}

    \hypertarget{tsagaan-sar-debt}{%
\section{Tsagaan Sar Debt}\label{tsagaan-sar-debt}}

\hypertarget{hypothesis-people-get-more-loans-right-before-tsagaan-sar.}{%
\subsection{Hypothesis: People get more loans right before Tsagaan
Sar.}\label{hypothesis-people-get-more-loans-right-before-tsagaan-sar.}}

    \hypertarget{data-mongol-bank}{%
\subsection{Data: Mongol Bank}\label{data-mongol-bank}}

\hypertarget{tools-python-and-facebooks-prophet}{%
\subsection{Tools: Python and Facebook's
Prophet}\label{tools-python-and-facebooks-prophet}}

    \hypertarget{full-code-to-produce-results}{%
\subsubsection{Full code to produce
results}\label{full-code-to-produce-results}}

    \begin{Verbatim}[commandchars=\\\{\}]
{\color{incolor}In [{\color{incolor}2}]:} \PY{c+c1}{\PYZsh{}Import required libraries}
        \PY{k+kn}{import} \PY{n+nn}{pandas} \PY{k}{as} \PY{n+nn}{pd}
        \PY{k+kn}{from} \PY{n+nn}{fbprophet} \PY{k}{import} \PY{n}{Prophet} 
        \PY{k+kn}{import} \PY{n+nn}{matplotlib}\PY{n+nn}{.}\PY{n+nn}{pyplot} \PY{k}{as} \PY{n+nn}{plt}
        \PY{o}{\PYZpc{}}\PY{k}{matplotlib} inline
        
        \PY{c+c1}{\PYZsh{}Load data}
        \PY{n}{df} \PY{o}{=} \PY{n}{pd}\PY{o}{.}\PY{n}{read\PYZus{}csv}\PY{p}{(}\PY{l+s+s1}{\PYZsq{}}\PY{l+s+s1}{data/loans.csv}\PY{l+s+s1}{\PYZsq{}}\PY{p}{,} \PY{n}{header}\PY{o}{=}\PY{l+m+mi}{0}\PY{p}{,} \PY{n}{names}\PY{o}{=}\PY{p}{[}\PY{l+s+s1}{\PYZsq{}}\PY{l+s+s1}{date}\PY{l+s+s1}{\PYZsq{}}\PY{p}{,}\PY{l+s+s1}{\PYZsq{}}\PY{l+s+s1}{salary}\PY{l+s+s1}{\PYZsq{}}\PY{p}{,}\PY{l+s+s1}{\PYZsq{}}\PY{l+s+s1}{pension}\PY{l+s+s1}{\PYZsq{}}\PY{p}{]}\PY{p}{)}
\end{Verbatim}

    \begin{Verbatim}[commandchars=\\\{\}]
{\color{incolor}In [{\color{incolor}3}]:} \PY{n}{df}\PY{o}{.}\PY{n}{head}\PY{p}{(}\PY{l+m+mi}{10}\PY{p}{)}
\end{Verbatim}

\begin{Verbatim}[commandchars=\\\{\}]
{\color{outcolor}Out[{\color{outcolor}3}]:}          date        salary      pension
        0  12/31/2008   37547.56092  43616.02673
        1   3/31/2009   30640.04101  55241.52788
        2   6/30/2009   43947.97785  46022.47222
        3   9/30/2009   76677.81398  52476.63136
        4  12/31/2009   51307.13591  43371.51284
        5   3/31/2010   58950.04688  52612.07351
        6   6/30/2010   92872.42111  46837.86520
        7   9/30/2010  116903.32770  50185.77887
        8  12/31/2010  111685.94790  62613.92135
        9   3/31/2011  208644.84980  82509.17204
\end{Verbatim}
            
    \begin{Verbatim}[commandchars=\\\{\}]
{\color{incolor}In [{\color{incolor}4}]:} \PY{n}{df}\PY{p}{[}\PY{l+s+s1}{\PYZsq{}}\PY{l+s+s1}{date}\PY{l+s+s1}{\PYZsq{}}\PY{p}{]} \PY{o}{=} \PY{n}{pd}\PY{o}{.}\PY{n}{to\PYZus{}datetime}\PY{p}{(}\PY{n}{df}\PY{p}{[}\PY{l+s+s1}{\PYZsq{}}\PY{l+s+s1}{date}\PY{l+s+s1}{\PYZsq{}}\PY{p}{]}\PY{p}{)}
        \PY{n}{pension} \PY{o}{=} \PY{n}{df}\PY{p}{[}\PY{p}{[}\PY{l+s+s1}{\PYZsq{}}\PY{l+s+s1}{date}\PY{l+s+s1}{\PYZsq{}}\PY{p}{,}\PY{l+s+s1}{\PYZsq{}}\PY{l+s+s1}{pension}\PY{l+s+s1}{\PYZsq{}}\PY{p}{]}\PY{p}{]}\PY{o}{.}\PY{n}{rename}\PY{p}{(}\PY{n}{columns}\PY{o}{=}\PY{p}{\PYZob{}}\PY{l+s+s1}{\PYZsq{}}\PY{l+s+s1}{date}\PY{l+s+s1}{\PYZsq{}}\PY{p}{:}\PY{l+s+s1}{\PYZsq{}}\PY{l+s+s1}{ds}\PY{l+s+s1}{\PYZsq{}}\PY{p}{,}\PY{l+s+s1}{\PYZsq{}}\PY{l+s+s1}{pension}\PY{l+s+s1}{\PYZsq{}}\PY{p}{:}\PY{l+s+s1}{\PYZsq{}}\PY{l+s+s1}{y}\PY{l+s+s1}{\PYZsq{}}\PY{p}{\PYZcb{}}\PY{p}{)}
\end{Verbatim}

    \hypertarget{fitting-a-prophet-model}{%
\subsubsection{Fitting a Prophet Model}\label{fitting-a-prophet-model}}

    \begin{Verbatim}[commandchars=\\\{\}]
{\color{incolor}In [{\color{incolor}5}]:} \PY{n}{m} \PY{o}{=} \PY{n}{Prophet}\PY{p}{(}\PY{n}{yearly\PYZus{}seasonality}\PY{o}{=}\PY{l+m+mi}{4}\PY{p}{)}
        \PY{n}{m}\PY{o}{.}\PY{n}{add\PYZus{}seasonality}\PY{p}{(}\PY{n}{name}\PY{o}{=}\PY{l+s+s1}{\PYZsq{}}\PY{l+s+s1}{quarterly}\PY{l+s+s1}{\PYZsq{}}\PY{p}{,} \PY{n}{period}\PY{o}{=}\PY{l+m+mf}{30.5}\PY{p}{,} \PY{n}{fourier\PYZus{}order}\PY{o}{=}\PY{l+m+mi}{5}\PY{p}{)}
        \PY{n}{m}\PY{o}{.}\PY{n}{fit}\PY{p}{(}\PY{n}{pension}\PY{p}{)}
\end{Verbatim}

    \begin{Verbatim}[commandchars=\\\{\}]
C:\textbackslash{}Users\textbackslash{}rober\textbackslash{}Anaconda3\textbackslash{}envs\textbackslash{}standard\textbackslash{}lib\textbackslash{}site-packages\textbackslash{}fbprophet\textbackslash{}forecaster.py:880: FutureWarning: Series.nonzero() is deprecated and will be removed in a future version.Use Series.to\_numpy().nonzero() instead
  min\_dt = dt.iloc[dt.nonzero()[0]].min()
INFO:fbprophet:Disabling weekly seasonality. Run prophet with weekly\_seasonality=True to override this.
INFO:fbprophet:Disabling daily seasonality. Run prophet with daily\_seasonality=True to override this.
C:\textbackslash{}Users\textbackslash{}rober\textbackslash{}Anaconda3\textbackslash{}envs\textbackslash{}standard\textbackslash{}lib\textbackslash{}site-packages\textbackslash{}pystan\textbackslash{}misc.py:399: FutureWarning: Conversion of the second argument of issubdtype from `float` to `np.floating` is deprecated. In future, it will be treated as `np.float64 == np.dtype(float).type`.
  elif np.issubdtype(np.asarray(v).dtype, float):

    \end{Verbatim}

\begin{Verbatim}[commandchars=\\\{\}]
{\color{outcolor}Out[{\color{outcolor}5}]:} <fbprophet.forecaster.Prophet at 0x2149e75c9b0>
\end{Verbatim}
            
    \begin{Verbatim}[commandchars=\\\{\}]
{\color{incolor}In [{\color{incolor}6}]:} \PY{c+c1}{\PYZsh{} Make future periods}
        \PY{n}{future} \PY{o}{=} \PY{n}{m}\PY{o}{.}\PY{n}{make\PYZus{}future\PYZus{}dataframe}\PY{p}{(}\PY{n}{periods}\PY{o}{=}\PY{l+m+mi}{8}\PY{p}{,} \PY{n}{freq}\PY{o}{=}\PY{l+s+s1}{\PYZsq{}}\PY{l+s+s1}{Q}\PY{l+s+s1}{\PYZsq{}}\PY{p}{)}
        
        \PY{c+c1}{\PYZsh{} Make forecast from fitted model}
        \PY{n}{forecast} \PY{o}{=} \PY{n}{m}\PY{o}{.}\PY{n}{predict}\PY{p}{(}\PY{n}{future}\PY{p}{)}
        \PY{n}{forecast}\PY{p}{[}\PY{p}{[}\PY{l+s+s1}{\PYZsq{}}\PY{l+s+s1}{ds}\PY{l+s+s1}{\PYZsq{}}\PY{p}{,} \PY{l+s+s1}{\PYZsq{}}\PY{l+s+s1}{yhat}\PY{l+s+s1}{\PYZsq{}}\PY{p}{,} \PY{l+s+s1}{\PYZsq{}}\PY{l+s+s1}{yhat\PYZus{}lower}\PY{l+s+s1}{\PYZsq{}}\PY{p}{,} \PY{l+s+s1}{\PYZsq{}}\PY{l+s+s1}{yhat\PYZus{}upper}\PY{l+s+s1}{\PYZsq{}}\PY{p}{]}\PY{p}{]}\PY{o}{.}\PY{n}{tail}\PY{p}{(}\PY{p}{)}
\end{Verbatim}

\begin{Verbatim}[commandchars=\\\{\}]
{\color{outcolor}Out[{\color{outcolor}6}]:}            ds           yhat     yhat\_lower     yhat\_upper
        43 2019-09-30  317393.418488  289992.885495  347549.374397
        44 2019-12-31  300926.311152  272304.755176  330223.397244
        45 2020-03-31  353513.007013  323609.687335  382599.552508
        46 2020-06-30  335064.998330  307512.211872  364689.373611
        47 2020-09-30  345002.350752  317498.907990  372318.799900
\end{Verbatim}
            
    \begin{Verbatim}[commandchars=\\\{\}]
{\color{incolor}In [{\color{incolor}7}]:} \PY{n}{fig1} \PY{o}{=} \PY{n}{m}\PY{o}{.}\PY{n}{plot}\PY{p}{(}\PY{n}{forecast}\PY{p}{)}
\end{Verbatim}

    \begin{Verbatim}[commandchars=\\\{\}]
C:\textbackslash{}Users\textbackslash{}rober\textbackslash{}Anaconda3\textbackslash{}envs\textbackslash{}standard\textbackslash{}lib\textbackslash{}site-packages\textbackslash{}pandas\textbackslash{}plotting\textbackslash{}\_converter.py:129: FutureWarning: Using an implicitly registered datetime converter for a matplotlib plotting method. The converter was registered by pandas on import. Future versions of pandas will require you to explicitly register matplotlib converters.

To register the converters:
	>>> from pandas.plotting import register\_matplotlib\_converters
	>>> register\_matplotlib\_converters()
  warnings.warn(msg, FutureWarning)

    \end{Verbatim}

    \begin{center}
    \adjustimage{max size={0.9\linewidth}{0.9\paperheight}}{output_53_1.png}
    \end{center}
    { \hspace*{\fill} \\}
    
    \begin{Verbatim}[commandchars=\\\{\}]
{\color{incolor}In [{\color{incolor}9}]:} \PY{k+kn}{from} \PY{n+nn}{fbprophet}\PY{n+nn}{.}\PY{n+nn}{plot} \PY{k}{import} \PY{n}{plot\PYZus{}yearly}
        \PY{n}{b} \PY{o}{=} \PY{n}{plot\PYZus{}yearly}\PY{p}{(}\PY{n}{m}\PY{p}{)}
        \PY{n}{plt}\PY{o}{.}\PY{n}{text}\PY{p}{(}\PY{n}{x} \PY{o}{=} \PY{l+m+mi}{736400}\PY{p}{,} \PY{n}{y} \PY{o}{=} \PY{l+m+mi}{20000}\PY{p}{,} \PY{n}{s} \PY{o}{=} \PY{l+s+s1}{\PYZsq{}}\PY{l+s+s1}{Pension Loan Cycle}\PY{l+s+s1}{\PYZsq{}}\PY{p}{,} \PY{n}{fontsize}\PY{o}{=}\PY{l+m+mi}{26}\PY{p}{,} \PY{n}{alpha}\PY{o}{=}\PY{o}{.}\PY{l+m+mi}{8}\PY{p}{)}
\end{Verbatim}

\begin{Verbatim}[commandchars=\\\{\}]
{\color{outcolor}Out[{\color{outcolor}9}]:} Text(736400, 20000, 'Pension Loan Cycle')
\end{Verbatim}
            
    \begin{center}
    \adjustimage{max size={0.9\linewidth}{0.9\paperheight}}{output_54_1.png}
    \end{center}
    { \hspace*{\fill} \\}
    
    \hypertarget{tools-for-everyone}{%
\section{Tools for everyone}\label{tools-for-everyone}}

\begin{itemize}
\tightlist
\item
  Data Illustrator, data-illustrator.com - FREE
\item
  Tableau Public is FREE for all
\item
  Google Data Studio, out of beta - FREE
\end{itemize}

    \hypertarget{advanced-analysis-tools}{%
\subsection{Advanced Analysis Tools}\label{advanced-analysis-tools}}

\begin{itemize}
\tightlist
\item
  D3
\item
  Pandas, Numpy (Python)
\item
  Dplyr, tidyr (R)
\item
  Matplotlib, Seaborn, Bokeh, Plotly
\end{itemize}

    \hypertarget{web-skills}{%
\subsection{Web Skills}\label{web-skills}}

\begin{itemize}
\tightlist
\item
  HTML
\item
  CSS
\item
  Javascript
\item
  Django
\end{itemize}

    \hypertarget{how-do-i-know-your-results-are-valid}{%
\section{How do I know your results are
valid?}\label{how-do-i-know-your-results-are-valid}}

\begin{itemize}
\tightlist
\item
  Have you tried to reproduce the results of a research paper?
\item
  Can you even get the data?
\item
  If you can't reproduce the results is the paper valid?
\end{itemize}

    \hypertarget{jupyter-notebooks}{%
\section{Jupyter Notebooks}\label{jupyter-notebooks}}

``Open-source web application that allows you to create and share
documents that contain live code, equations, visualizations and
narrative text.''\\
-Jupyter.org

\begin{itemize}
\tightlist
\item
  \#1 Data Science Environment
\item
  Supports many programming languages (including R and Python)
\item
  Your entire research paper can live in a notebook.
\end{itemize}

    \hypertarget{reproducibility}{%
\subsection{Reproducibility}\label{reproducibility}}

\textbf{Can you retrace every step you took to create your paper?} - End
to end research workflow - Write, visualize, and analyze in one place -
Be able to prove each step of your analysis - Easily share your work
with others over Github or elsewhere

    Import scraped crime data from http://crimemap.police.gov.mn and store
to crimes dataframe.

    \begin{Verbatim}[commandchars=\\\{\}]
{\color{incolor}In [{\color{incolor}19}]:} \PY{k+kn}{import} \PY{n+nn}{pandas} \PY{k}{as} \PY{n+nn}{pd}
         \PY{n}{crimes} \PY{o}{=} \PY{n}{pd}\PY{o}{.}\PY{n}{read\PYZus{}excel}\PY{p}{(}\PY{l+s+s2}{\PYZdq{}}\PY{l+s+s2}{assets/Mongolia Crime Data Cleaned.xlsx}\PY{l+s+s2}{\PYZdq{}}\PY{p}{)}
\end{Verbatim}

    \begin{Verbatim}[commandchars=\\\{\}]
{\color{incolor}In [{\color{incolor}20}]:} \PY{c+c1}{\PYZsh{} Take a look at head of dataframe to understand the features we have.}
         \PY{n}{crimes}\PY{o}{.}\PY{n}{head}\PY{p}{(}\PY{p}{)}
\end{Verbatim}

\begin{Verbatim}[commandchars=\\\{\}]
{\color{outcolor}Out[{\color{outcolor}20}]:}                     Гэмт хэргийн төрөл  Хэргийн дугаар Хэргийн огноо  \textbackslash{}
         0  Бусдын бие махбодид гэмтэл учруулах    2.014260e+12    2014-07-01   
         1  Бусдын бие махбодид гэмтэл учруулах    2.014260e+12    2014-07-01   
         2  Бусдын бие махбодид гэмтэл учруулах    2.014260e+12    2014-07-01   
         3  Бусдын бие махбодид гэмтэл учруулах    2.014260e+12    2014-07-01   
         4             Хулгайлах - Иргэдийн өмч    2.014250e+12    2014-07-01   
         
                                   Хэргийн байршил    Хот/Аймаг            Дүүрэг  \textbackslash{}
         0        Хан-Уул, 15-р хороо, жапан таун,  Улаанбаатар           Хан-Уул   
         1              Баянгол дүүрэг, 1-р хороо,  Улаанбаатар    Баянгол дүүрэг   
         2                    Хан-Уул, 10-р хороо,  Улаанбаатар           Хан-Уул   
         3             Баянгол дүүрэг, 17-р хороо,  Улаанбаатар    Баянгол дүүрэг   
         4  Чингэлтэй дүүрэг, 12-р хороо, булгийн,  Улаанбаатар  Чингэлтэй дүүрэг   
         
             Хороо/Сум                                 Шалгасан ЦХ  
         0  15-р хороо    Хан-Уул дүүрэг дэх цагдаагийн 1-р хэлтэс  
         1   1-р хороо    Баянгол дүүрэг дэх цагдаагийн 2-р хэлтэс  
         2  10-р хороо    Хан-Уул дүүрэг дэх цагдаагийн 2-р хэлтэс  
         3  17-р хороо    Баянгол дүүрэг дэх цагдаагийн 1-р хэлтэс  
         4  12-р хороо  Чингэлтэй дүүрэг дэх цагдаагийн 2-р хэлтэс  
\end{Verbatim}
            
    What are the unique crime categories? We are looking for ``murder'' or
``intentional homicide''.

    \begin{Verbatim}[commandchars=\\\{\}]
{\color{incolor}In [{\color{incolor}21}]:} \PY{n}{crimes}\PY{p}{[}\PY{l+s+s1}{\PYZsq{}}\PY{l+s+s1}{Гэмт хэргийн төрөл}\PY{l+s+s1}{\PYZsq{}}\PY{p}{]}\PY{o}{.}\PY{n}{unique}\PY{p}{(}\PY{p}{)}
\end{Verbatim}

\begin{Verbatim}[commandchars=\\\{\}]
{\color{outcolor}Out[{\color{outcolor}21}]:} array(['Бусдын бие махбодид гэмтэл учруулах', 'Хулгайлах - Иргэдийн өмч',
                'ТХХАББАЖЗ', 'Хулгайлах - Тээврийн хэрэгсэл', 'Танхайрах',
                'Булаах', 'Хулгайлах - Мал', 'Хүчиндэх', 'Дээрэмдэх',
                'Хулгайлах - Халаас', 'Хүнийг санаатай алах', nan], dtype=object)
\end{Verbatim}
            
    We found the murder category. It is called ``Хүнийг санаатай алах''. How
many murders were there in 2016?

    \begin{Verbatim}[commandchars=\\\{\}]
{\color{incolor}In [{\color{incolor}66}]:} \PY{n}{murder\PYZus{}2016} \PY{o}{=} \PY{n}{crimes}\PY{p}{[}\PY{p}{(}\PY{n}{crimes}\PY{p}{[}\PY{l+s+s1}{\PYZsq{}}\PY{l+s+s1}{Гэмт хэргийн төрөл}\PY{l+s+s1}{\PYZsq{}}\PY{p}{]} \PY{o}{==} \PY{l+s+s1}{\PYZsq{}}\PY{l+s+s1}{Хүнийг санаатай алах}\PY{l+s+s1}{\PYZsq{}}\PY{p}{)} \PY{o}{\PYZam{}} \PY{p}{(}\PY{n}{crimes}\PY{p}{[}\PY{l+s+s1}{\PYZsq{}}\PY{l+s+s1}{Хэргийн огноо}\PY{l+s+s1}{\PYZsq{}}\PY{p}{]}\PY{o}{.}\PY{n}{dt}\PY{o}{.}\PY{n}{year} \PY{o}{==} \PY{l+m+mi}{2016}\PY{p}{)}\PY{p}{]}
         \PY{n}{murder\PYZus{}2016} \PY{o}{=} \PY{n}{murder\PYZus{}2016}\PY{o}{.}\PY{n}{reset\PYZus{}index}\PY{p}{(}\PY{n}{drop}\PY{o}{=}\PY{k+kc}{True}\PY{p}{)}
         \PY{n+nb}{print}\PY{p}{(}\PY{l+s+s2}{\PYZdq{}}\PY{l+s+s2}{There were}\PY{l+s+s2}{\PYZdq{}}\PY{p}{,} \PY{n+nb}{len}\PY{p}{(}\PY{n}{murder\PYZus{}2016}\PY{o}{.}\PY{n}{index}\PY{p}{)}\PY{p}{,} \PY{l+s+s2}{\PYZdq{}}\PY{l+s+s2}{murders in Mongolia in 2016.}\PY{l+s+s2}{\PYZdq{}}\PY{p}{)}
\end{Verbatim}

    \begin{Verbatim}[commandchars=\\\{\}]
There were 173 murders in Mongolia in 2016.

    \end{Verbatim}

    \hypertarget{your-colleague-says-a-visual-showing-murders-each-month-will-have-more-impact.}{%
\subsection{Your colleague says, ``A visual showing murders each month
will have more
impact.''}\label{your-colleague-says-a-visual-showing-murders-each-month-will-have-more-impact.}}

    Find the number of murders each month.

    \begin{Verbatim}[commandchars=\\\{\}]
{\color{incolor}In [{\color{incolor}85}]:} \PY{n}{murder\PYZus{}2016}\PY{p}{[}\PY{l+s+s1}{\PYZsq{}}\PY{l+s+s1}{Хэргийн дугаар}\PY{l+s+s1}{\PYZsq{}}\PY{p}{]}\PY{o}{.}\PY{n}{groupby}\PY{p}{(}\PY{n}{murder\PYZus{}2016}\PY{p}{[}\PY{l+s+s1}{\PYZsq{}}\PY{l+s+s1}{Хэргийн огноо}\PY{l+s+s1}{\PYZsq{}}\PY{p}{]}\PY{o}{.}\PY{n}{dt}\PY{o}{.}\PY{n}{month}\PY{p}{)}\PY{o}{.}\PY{n}{agg}\PY{p}{(}\PY{l+s+s1}{\PYZsq{}}\PY{l+s+s1}{count}\PY{l+s+s1}{\PYZsq{}}\PY{p}{)}
\end{Verbatim}

\begin{Verbatim}[commandchars=\\\{\}]
{\color{outcolor}Out[{\color{outcolor}85}]:} Хэргийн огноо
         1     12
         2     13
         3     11
         4     12
         5     16
         6     12
         7     17
         8     13
         9     11
         10    19
         11    18
         12    19
         Name: Хэргийн дугаар, dtype: int64
\end{Verbatim}
            
    Import Matplotlib and use magic to show plots inline

    \begin{Verbatim}[commandchars=\\\{\}]
{\color{incolor}In [{\color{incolor}93}]:} \PY{k+kn}{import} \PY{n+nn}{matplotlib}\PY{n+nn}{.}\PY{n+nn}{pyplot} \PY{k}{as} \PY{n+nn}{plt}
         \PY{o}{\PYZpc{}}\PY{k}{matplotlib} inline
\end{Verbatim}

    Assign month names to ``x'' values and murder totals for each month to
``y'' values.

    \begin{Verbatim}[commandchars=\\\{\}]
{\color{incolor}In [{\color{incolor}94}]:} \PY{n}{months} \PY{o}{=} \PY{p}{[}\PY{l+s+s1}{\PYZsq{}}\PY{l+s+s1}{Jan}\PY{l+s+s1}{\PYZsq{}}\PY{p}{,} \PY{l+s+s1}{\PYZsq{}}\PY{l+s+s1}{Feb}\PY{l+s+s1}{\PYZsq{}}\PY{p}{,} \PY{l+s+s1}{\PYZsq{}}\PY{l+s+s1}{Mar}\PY{l+s+s1}{\PYZsq{}}\PY{p}{,} \PY{l+s+s1}{\PYZsq{}}\PY{l+s+s1}{Apr}\PY{l+s+s1}{\PYZsq{}}\PY{p}{,} \PY{l+s+s1}{\PYZsq{}}\PY{l+s+s1}{May}\PY{l+s+s1}{\PYZsq{}}\PY{p}{,} \PY{l+s+s1}{\PYZsq{}}\PY{l+s+s1}{Jun}\PY{l+s+s1}{\PYZsq{}}\PY{p}{,} \PY{l+s+s1}{\PYZsq{}}\PY{l+s+s1}{Jul}\PY{l+s+s1}{\PYZsq{}}\PY{p}{,} \PY{l+s+s1}{\PYZsq{}}\PY{l+s+s1}{Aug}\PY{l+s+s1}{\PYZsq{}}\PY{p}{,} \PY{l+s+s1}{\PYZsq{}}\PY{l+s+s1}{Sep}\PY{l+s+s1}{\PYZsq{}}\PY{p}{,} \PY{l+s+s1}{\PYZsq{}}\PY{l+s+s1}{Oct}\PY{l+s+s1}{\PYZsq{}}\PY{p}{,} \PY{l+s+s1}{\PYZsq{}}\PY{l+s+s1}{Nov}\PY{l+s+s1}{\PYZsq{}}\PY{p}{,} \PY{l+s+s1}{\PYZsq{}}\PY{l+s+s1}{Dec}\PY{l+s+s1}{\PYZsq{}}\PY{p}{]}
         \PY{n}{y} \PY{o}{=} \PY{n}{murder\PYZus{}2016}\PY{p}{[}\PY{l+s+s1}{\PYZsq{}}\PY{l+s+s1}{Хэргийн дугаар}\PY{l+s+s1}{\PYZsq{}}\PY{p}{]}\PY{o}{.}\PY{n}{groupby}\PY{p}{(}\PY{n}{murder\PYZus{}2016}\PY{p}{[}\PY{l+s+s1}{\PYZsq{}}\PY{l+s+s1}{Хэргийн огноо}\PY{l+s+s1}{\PYZsq{}}\PY{p}{]}\PY{o}{.}\PY{n}{dt}\PY{o}{.}\PY{n}{month}\PY{p}{)}\PY{o}{.}\PY{n}{agg}\PY{p}{(}\PY{l+s+s1}{\PYZsq{}}\PY{l+s+s1}{count}\PY{l+s+s1}{\PYZsq{}}\PY{p}{)}
\end{Verbatim}

    Build the plot and assign titles and labels.

    \begin{Verbatim}[commandchars=\\\{\}]
{\color{incolor}In [{\color{incolor}95}]:} \PY{n}{plt}\PY{o}{.}\PY{n}{figure}\PY{p}{(}\PY{n}{figsize}\PY{o}{=}\PY{p}{(}\PY{l+m+mi}{8}\PY{p}{,}\PY{l+m+mi}{5}\PY{p}{)}\PY{p}{)}
         \PY{n}{plt}\PY{o}{.}\PY{n}{title}\PY{p}{(}\PY{l+s+s2}{\PYZdq{}}\PY{l+s+s2}{2016 Murders by Month}\PY{l+s+s2}{\PYZdq{}}\PY{p}{)}
         \PY{n}{plt}\PY{o}{.}\PY{n}{xlabel}\PY{p}{(}\PY{l+s+s2}{\PYZdq{}}\PY{l+s+s2}{Month}\PY{l+s+s2}{\PYZdq{}}\PY{p}{)}
         \PY{n}{plt}\PY{o}{.}\PY{n}{ylabel}\PY{p}{(}\PY{l+s+s2}{\PYZdq{}}\PY{l+s+s2}{Murders}\PY{l+s+s2}{\PYZdq{}}\PY{p}{)}
         \PY{n}{plt}\PY{o}{.}\PY{n}{bar}\PY{p}{(}\PY{n}{months}\PY{p}{,} \PY{n}{y}\PY{p}{)}
\end{Verbatim}

\begin{Verbatim}[commandchars=\\\{\}]
{\color{outcolor}Out[{\color{outcolor}95}]:} <BarContainer object of 12 artists>
\end{Verbatim}
            
    \begin{center}
    \adjustimage{max size={0.9\linewidth}{0.9\paperheight}}{output_76_1.png}
    \end{center}
    { \hspace*{\fill} \\}
    
    \hypertarget{i-encourage-each-of-you-to-learn-to-use-jupyter-notebooks-and-learn-python-or-r}{%
\subsection{I encourage each of you to learn to use Jupyter Notebooks
and learn Python or
R}\label{i-encourage-each-of-you-to-learn-to-use-jupyter-notebooks-and-learn-python-or-r}}

\textbf{Resources} - Anaconda.com (home to the best Jupyter
distribution) - Datacamp.com (Python and R training) -
Data-illustrator.com - Datastudio.google.com - Scrapinghub.com (Portia,
Scrapy)

    \hypertarget{thank-you.}{%
\section{Thank you.}\label{thank-you.}}

\hypertarget{questions}{%
\subsection{Questions?}\label{questions}}


    % Add a bibliography block to the postdoc
    
    
    
    \end{document}
